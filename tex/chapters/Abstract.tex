\begin{abstract}

    Man's curiosity has been the drive that pushes innovation and progress since
    the beginning of research. In order to satisfy this curiosity, Man keeps on
    finding ways to observe and measure the environment around him. So intense
    is this drive that it evolved into science. Researchers collect information
    from further and harder to reach areas. One area of this attempt makes use
    of wireless sensor networks. These sensors are able to collect usefull data
    and pass them to the researchers over a wireless interface therefore making
    the collection easier. However the dynamic nature of these networks proves
    that connectivity of these sensors is a serious challenge. Another need that
    arises from the nature of the problem of information gathering from
    disconnected areas is the optimal use of energy by minimizing the
    consumption. As a result the state of the art The state of the art has been
    swapping from connected mesh networks to distributed networks.

    Distrubuted networks that are often disconnected require an active node to
    move between the different clusters and collect the information that has
    been gathered. In order to achieve this goal often Unmanned Aerial Vehicles
    are put to use. Consequently, In the recent years, we can see a rise in
    interest in the area of of UAVs. The versatility of UAVs and their unique
    ability to reach areas that the humans are unable to go themselves brings
    them to the bleeding edge of innovative technologies. Right now we are able
    to utilise wireless technologies to observe ocean surface, the behaviour of
    animals or the inside of volcanoes in an automated manner [citation needed].

    Unmanned Aerial Vehicles are more and more commonly used and promise not
    only to collect information for us but also to deliver data or parcels
    [amazon drones]. While UAVs are being deployed in such different areas it is
    important to understand their limitations. As a result people have been
    developing and evaluating UAV platforms. The cost of these platforms is far
    from trivial. This high cost makes real life experimentation with UAVs cubbersome as
    it has increased risk. Simulation provides a solution to the problem as it
    enables the study of the behaviour of UAVs reducing the risk involved.

    Sharing this research interest, this project attempts to explore the area of
    Path Planning in distributed wireless sensor networks where each
    The proposed project attempts to explore the area of Path Planning. We take
    into consideration that multiple UAVs can be used to collect information. We

    They have limited resources on their hull and they need to be utilised
    fully. One of the most important questions that rises from their broad
    deployment is how to extend their flight times. The above show the need for
    research in the area of Path Planning with UAVs

\end{abstract}
