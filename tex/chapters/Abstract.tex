\begin{abstract}

    Man's curiosity has been the drive that pushes innovation and progress since
    the beginning of research. In order to satisfy this curiosity, Man keeps on
    finding ways to observe and measure the environment around him. So intense
    is this drive that it evolved into science. The state of the art has been
    swapping from connected mesh networks to distributed networks.

    In the recent years, we can see a rise in interest in the area of of
    Unmanned Aerial Vehicles (UAVs). The versatility UAVs and their unique
    ability to reach areas that the humans are unable to go themselves brings
    them to the bleeding edge of innovative technologies. Right now we are able
    to utilise wireless technologies to observe ocean surface, the behaviour of
    animals or the inside of volcanoes in an automated manner [citation needed].

    Unmanned Aerial Vehicles are more and more commonly used and promise not
    only to collect information for us but also to deliver data or parcels
    [amazon drones]. While UAVs are being deployed in such different areas it is
    important to understand their limitations. As a result people have been
    developing and evaluating UAV platforms. The cost of these platforms is far
    from trivial. Thus simulation of the behaviour of UAVs is required.

    The need for simulation is what

    The proposed project attempts to explore the area of Path Planning. We take
    into consideration that multiple UAVs can be used to collect information. We

    They have limited resources on their hull and they need to be utilised
    fully. One of the most important questions that rises from their broad
    deployment is how to extend their flight times. The above show the need for
    research in the area of Path Planning with UAVs



\end{abstract}
