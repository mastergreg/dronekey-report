\section{CRATES}
CRATES stands for "Cognitive Robotics Architecture for Tightly-Coupled
Experiments and Simulation" [Site at bitbucket (andrew)]. Crates is a simulation
framework for robot problems. It has two main goals, first to enable researchers
to create high-level controllers for platforms in a simulated environment.
Secondly CRATES aims to allow the transfer of these high-level controllers, in
real life experiments. It uses the ROS backbone as a messaging system and
provides a hardware abstraction layer (HAL) for the controllers. These
controllers interract through the ROS backbone which is also supported on the
real life platforms. As a result the same controllers that are used in
simulation can interract through the HAL both with the simulation engine and the
platforms. This provides a very helpful tool to researchers as no change is
required in the controllers themselves.


In order to provide the simulation environment, CRATES, makes use of the Gazebo
physics engine. It provides OpenGL accelerated visualisation of the simulation
in real time. The simulation is enriched with military wind models as well as
with the GPS toolkit to simulate global navigation satelite systems.
GeographicLib enables the projection between different coordinate frames. The
engine is enriched with noise models for the measurements in each controller. In
our project we aim to further extend CRATES and provide a radio model which can
be used for communication between two objects within the simulation.


