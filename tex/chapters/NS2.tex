\section{NS2}
NS2 is a discrete event simulator targeted at networking research. It interprets
scripts of network events at distinct times. These are written in the tcl
programming language in a plaintext file. NS2 has long been used in multiple
research projects and it has a large enough code base. Within ns2 there are
models for multiple network stacks. It provides support for wired and wireless
networks, different mac layers which provide a distinct versatility. However, it
also has distinct limitations with regard to dynamic window advertisement and
explicit congestion notification. Regardless it is a well known research
standard network simulator.

NS2 is by default a non interactive simulator. It parses the events script
beforehand and executes them at runtime. This is also one of it's limitations,
it is unable by default to handle dynamic topologies. Moreover it does not
support wireless simulation in the 3-dimensional space. The radio model used for
the propagation of the signal within ns2 is a flat rectangular area.

For our project we attempt to use it as an interactive simulator, keeping it in
sync with the environment simulator.

